% Options for packages loaded elsewhere
\PassOptionsToPackage{unicode}{hyperref}
\PassOptionsToPackage{hyphens}{url}
\documentclass[
]{article}
\usepackage{xcolor}
\usepackage{amsmath,amssymb}
\setcounter{secnumdepth}{-\maxdimen} % remove section numbering
\usepackage{iftex}
\ifPDFTeX
  \usepackage[T1]{fontenc}
  \usepackage[utf8]{inputenc}
  \usepackage{textcomp} % provide euro and other symbols
\else % if luatex or xetex
  \usepackage{unicode-math} % this also loads fontspec
  \defaultfontfeatures{Scale=MatchLowercase}
  \defaultfontfeatures[\rmfamily]{Ligatures=TeX,Scale=1}
\fi
\usepackage{lmodern}
\ifPDFTeX\else
  % xetex/luatex font selection
\fi
% Use upquote if available, for straight quotes in verbatim environments
\IfFileExists{upquote.sty}{\usepackage{upquote}}{}
\IfFileExists{microtype.sty}{% use microtype if available
  \usepackage[]{microtype}
  \UseMicrotypeSet[protrusion]{basicmath} % disable protrusion for tt fonts
}{}
\makeatletter
\@ifundefined{KOMAClassName}{% if non-KOMA class
  \IfFileExists{parskip.sty}{%
    \usepackage{parskip}
  }{% else
    \setlength{\parindent}{0pt}
    \setlength{\parskip}{6pt plus 2pt minus 1pt}}
}{% if KOMA class
  \KOMAoptions{parskip=half}}
\makeatother
\setlength{\emergencystretch}{3em} % prevent overfull lines
\providecommand{\tightlist}{%
  \setlength{\itemsep}{0pt}\setlength{\parskip}{0pt}}
% header.tex
% 用于 Pandoc 转 PDF 时自动加载 mhchem 宏包支持化学式
\usepackage[version=4]{mhchem}
\usepackage{xeCJK}
\usepackage{bookmark}
\IfFileExists{xurl.sty}{\usepackage{xurl}}{} % add URL line breaks if available
\urlstyle{same}
\hypersetup{
  hidelinks,
  pdfcreator={LaTeX via pandoc}}

\author{}
\date{}

\begin{document}

豆包:

\section{识别题目}\label{ux8bc6ux522bux9898ux76ee}

下列物质的俗名、化学式、分类完全正确的是( ~ ) A.
水银、\(\mathrm{H}\mathrm{g}\)、单质 B.
熟石灰、\(\mathrm{C}\mathrm{a}\mathrm{O}\)、氧化物 C.
苛性钠、\(\ce{NaOH}\)、盐 D.
纯碱、\(\mathrm{N}{\mathrm{a}}_{\mathrm{2}}\mathrm{C}{\mathrm{O}}_{\mathrm{3}}\)、碱

\begin{center}\rule{0.5\linewidth}{0.5pt}\end{center}

\section{答案}\label{ux7b54ux6848}

A

\begin{center}\rule{0.5\linewidth}{0.5pt}\end{center}

\section{💡解析}\label{ux89e3ux6790}

\subsection{逐项分析}\label{ux9010ux9879ux5206ux6790}

\paragraph{A 选项}\label{a-ux9009ux9879}

水银是汞这种单质的俗名,汞的化学式为(\mathrm{H}\mathrm{g})。单质是由同种元素组成的纯净物,汞只由汞元素组成,所以属于单质,A
正确。 \#\#\#\# B 选项
熟石灰是氢氧化钙的俗名,其化学式应该是(\ce{Ca(OH)_{2}}),而不是(\mathrm{C}\mathrm{a}\mathrm{O})。碱是指在水溶液中电离出的阴离子全部是(\ce{OH^{-}})的化合物,氢氧化钙在水溶液中会电离出(\ce{Ca^{2 + }})和(\ce{OH^{-}}),阴离子全部是(\ce{OH^{-}}),所以氢氧化钙属于碱,并非氧化物,B
错误。 \#\#\#\# C 选项
苛性钠是氢氧化钠的俗名,其化学式为(\ce{NaOH})。碱的定义是在水溶液中电离出的阴离子全部是(\ce{OH^{-}})的化合物,氢氧化钠在水溶液中电离出(\ce{Na^{ + }})和(\ce{OH^{-}}),阴离子全部是(\ce{OH^{-}}),所以它属于碱,而不是盐,盐是由金属离子(或铵根离子)与酸根离子组成的化合物,C
错误。 \#\#\#\# D 选项
纯碱是碳酸钠的俗名,化学式为(\ce{Na_{2}CO_{3}})。由金属离子((\ce{Na^{ + }}))和酸根离子((\ce{CO_{3}^{2 - }}))构成的化合物属于盐,并非碱,所以碳酸钠属于盐,D
错误。 \#\# 总结 故本题答案为:A。

\end{document}
